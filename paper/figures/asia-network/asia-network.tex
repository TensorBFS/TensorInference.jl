\documentclass[varwidth=14cm]{standalone}

\usepackage{tikz}

% Set fount used by JOSS papers
\renewcommand\familydefault{\sfdefault}

% \fbox tip:
% - https://tex.stackexchange.com/questions/119984/subfigures-side-by-side-with-captions
\usepackage{subcaption}

% tables generated by quarto-beamer use longtable
\usepackage{booktabs}
\usepackage{longtable}

\usetikzlibrary{arrows.meta}

\begin{document}

\tikzset{ 
  myarrow/.style= {-{Stealth[scale=1.0]},shorten >=2pt},
  myvar/.style={circle, thick, draw=black},
    table/.style={
        matrix of nodes,
        row sep=-\pgflinewidth,
        column sep=-\pgflinewidth,
        nodes={
            rectangle,
            draw=black,
            align=center
        },
        minimum height=1.5em,
        text depth=0.5ex,
        text height=2ex,
        nodes in empty cells,
%%
        every even row/.style={
            nodes={fill=gray!20}
        },
        column 1/.style={
            nodes={text width=2em,font=\bfseries}
        },
        row 1/.style={
            nodes={
                fill=black,
                text=white,
                font=\bfseries
            }
        }
    }
}

\begin{minipage}{0.4\textwidth}
  %\scalebox{0.60}{
    \begin{tikzpicture}[]
      \matrix[row sep=0.5cm,column sep=0.5cm] {
      % First line
      \node (a) [myvar] {$A$};  &
                                &
                                &
      \node (s) [myvar] {$S$};  &
                               \\
      % Second line
      \node (t) [myvar] {$T$};  &
                                &
      \node (l) [myvar] {$L$};  &
                                &
      \node (b) [myvar] {$B$}; \\
      % Third line
                                &
      \node (e) [myvar] {$E$};  &
                                &
                                &
                               \\
      % Forth line
      \node (x) [myvar] {$X$};  &
                                &
                                &
      \node (d) [myvar] {$D$};  &
                               \\
      };
      \draw [myarrow]  (a) edge (t);
      \draw [myarrow]  (s) edge (l);
      \draw [myarrow]  (s) edge (b);
      \draw [myarrow]  (t) edge (e);
      \draw [myarrow]  (l) edge (e);
      \draw [myarrow]  (e) edge (x);
      \draw [myarrow]  (e) edge (d);
      \draw [myarrow]  (b) edge (d);
    \end{tikzpicture}
  %}
\end{minipage}%
\begin{minipage}{0.6\textwidth}
  \begin{longtable}[]{@{}cl@{}}
    \toprule()
    \textbf{Random variable}    & \textbf{Meaning}                         \\
    \midrule()
    \endhead
    \texttt{A}                  & Recent trip to Asia                      \\
    \texttt{T}                  & Patient has tuberculosis                 \\
    \texttt{S}                  & Patient is a smoker                      \\
    \texttt{L}                  & Patient has lung cancer                  \\
    \texttt{B}                  & Patient has bronchitis                   \\
    \texttt{E}                  & Patient has \texttt{T} and/or \texttt{L} \\
    \texttt{X}                  & Chest X-Ray is positive                  \\
    \texttt{D}                  & Patient has dyspnoea                     \\
    \bottomrule()
  \end{longtable}
\end{minipage}

%\begin{figure}
%  \centering
%  \begin{subfigure}{.5\textwidth}
%    \centering
%    %\begin{tikzpicture}
%    %  \path[mindmap,concept color=myred,text=white]
%    %    node[concept] {}
%    %    [clockwise from=60]
%    %    child[concept color=mygreen] { 
%    %      node[concept, shift={(60:\d)}] {} 
%    %    }
%    %    child[concept color=mypurple, grow=east] { 
%    %      node[concept, shift={(0:\d)}] {}
%    %    }
%    %    ;
%    %\end{tikzpicture}
%    \caption{A subfigure}
%    \label{fig:sub1}
%  \end{subfigure}%
%  \begin{subfigure}{.5\textwidth}
%    \centering
%    %\begin{longtable}[]{@{}cl@{}}
%    %  \toprule()
%    %  \textbf{Random variable} & \textbf{Meaning} \\
%    %  \midrule()
%    %  \endhead
%    %  \texttt{A} & Recent trip to Asia \\
%    %  \texttt{T} & Patient has tuberculosis \\
%    %  \texttt{S} & Patient is a smoker \\
%    %  \texttt{L} & Patient has lung cancer \\
%    %  \texttt{B} & Patient has bronchitis \\
%    %  \texttt{E} & Patient has \texttt{T} and/or \texttt{L} \\
%    %  \texttt{X} & Chest X-Ray is positive \\
%    %  \texttt{D} & Patient has dyspnoea \\
%    %  \bottomrule()
%    %\end{longtable}
%    \caption{A subfigure}
%    \label{fig:sub2}
%  \end{subfigure}
%  \caption{A figure with two subfigures}
%  \label{fig:test}
%\end{figure}

\end{document}
